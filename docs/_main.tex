% Options for packages loaded elsewhere
\PassOptionsToPackage{unicode}{hyperref}
\PassOptionsToPackage{hyphens}{url}
\documentclass[
]{book}
\usepackage{xcolor}
\usepackage{amsmath,amssymb}
\setcounter{secnumdepth}{5}
\usepackage{iftex}
\ifPDFTeX
  \usepackage[T1]{fontenc}
  \usepackage[utf8]{inputenc}
  \usepackage{textcomp} % provide euro and other symbols
\else % if luatex or xetex
  \usepackage{unicode-math} % this also loads fontspec
  \defaultfontfeatures{Scale=MatchLowercase}
  \defaultfontfeatures[\rmfamily]{Ligatures=TeX,Scale=1}
\fi
\usepackage{lmodern}
\ifPDFTeX\else
  % xetex/luatex font selection
\fi
% Use upquote if available, for straight quotes in verbatim environments
\IfFileExists{upquote.sty}{\usepackage{upquote}}{}
\IfFileExists{microtype.sty}{% use microtype if available
  \usepackage[]{microtype}
  \UseMicrotypeSet[protrusion]{basicmath} % disable protrusion for tt fonts
}{}
\makeatletter
\@ifundefined{KOMAClassName}{% if non-KOMA class
  \IfFileExists{parskip.sty}{%
    \usepackage{parskip}
  }{% else
    \setlength{\parindent}{0pt}
    \setlength{\parskip}{6pt plus 2pt minus 1pt}}
}{% if KOMA class
  \KOMAoptions{parskip=half}}
\makeatother
\usepackage{color}
\usepackage{fancyvrb}
\newcommand{\VerbBar}{|}
\newcommand{\VERB}{\Verb[commandchars=\\\{\}]}
\DefineVerbatimEnvironment{Highlighting}{Verbatim}{commandchars=\\\{\}}
% Add ',fontsize=\small' for more characters per line
\usepackage{framed}
\definecolor{shadecolor}{RGB}{248,248,248}
\newenvironment{Shaded}{\begin{snugshade}}{\end{snugshade}}
\newcommand{\AlertTok}[1]{\textcolor[rgb]{0.94,0.16,0.16}{#1}}
\newcommand{\AnnotationTok}[1]{\textcolor[rgb]{0.56,0.35,0.01}{\textbf{\textit{#1}}}}
\newcommand{\AttributeTok}[1]{\textcolor[rgb]{0.13,0.29,0.53}{#1}}
\newcommand{\BaseNTok}[1]{\textcolor[rgb]{0.00,0.00,0.81}{#1}}
\newcommand{\BuiltInTok}[1]{#1}
\newcommand{\CharTok}[1]{\textcolor[rgb]{0.31,0.60,0.02}{#1}}
\newcommand{\CommentTok}[1]{\textcolor[rgb]{0.56,0.35,0.01}{\textit{#1}}}
\newcommand{\CommentVarTok}[1]{\textcolor[rgb]{0.56,0.35,0.01}{\textbf{\textit{#1}}}}
\newcommand{\ConstantTok}[1]{\textcolor[rgb]{0.56,0.35,0.01}{#1}}
\newcommand{\ControlFlowTok}[1]{\textcolor[rgb]{0.13,0.29,0.53}{\textbf{#1}}}
\newcommand{\DataTypeTok}[1]{\textcolor[rgb]{0.13,0.29,0.53}{#1}}
\newcommand{\DecValTok}[1]{\textcolor[rgb]{0.00,0.00,0.81}{#1}}
\newcommand{\DocumentationTok}[1]{\textcolor[rgb]{0.56,0.35,0.01}{\textbf{\textit{#1}}}}
\newcommand{\ErrorTok}[1]{\textcolor[rgb]{0.64,0.00,0.00}{\textbf{#1}}}
\newcommand{\ExtensionTok}[1]{#1}
\newcommand{\FloatTok}[1]{\textcolor[rgb]{0.00,0.00,0.81}{#1}}
\newcommand{\FunctionTok}[1]{\textcolor[rgb]{0.13,0.29,0.53}{\textbf{#1}}}
\newcommand{\ImportTok}[1]{#1}
\newcommand{\InformationTok}[1]{\textcolor[rgb]{0.56,0.35,0.01}{\textbf{\textit{#1}}}}
\newcommand{\KeywordTok}[1]{\textcolor[rgb]{0.13,0.29,0.53}{\textbf{#1}}}
\newcommand{\NormalTok}[1]{#1}
\newcommand{\OperatorTok}[1]{\textcolor[rgb]{0.81,0.36,0.00}{\textbf{#1}}}
\newcommand{\OtherTok}[1]{\textcolor[rgb]{0.56,0.35,0.01}{#1}}
\newcommand{\PreprocessorTok}[1]{\textcolor[rgb]{0.56,0.35,0.01}{\textit{#1}}}
\newcommand{\RegionMarkerTok}[1]{#1}
\newcommand{\SpecialCharTok}[1]{\textcolor[rgb]{0.81,0.36,0.00}{\textbf{#1}}}
\newcommand{\SpecialStringTok}[1]{\textcolor[rgb]{0.31,0.60,0.02}{#1}}
\newcommand{\StringTok}[1]{\textcolor[rgb]{0.31,0.60,0.02}{#1}}
\newcommand{\VariableTok}[1]{\textcolor[rgb]{0.00,0.00,0.00}{#1}}
\newcommand{\VerbatimStringTok}[1]{\textcolor[rgb]{0.31,0.60,0.02}{#1}}
\newcommand{\WarningTok}[1]{\textcolor[rgb]{0.56,0.35,0.01}{\textbf{\textit{#1}}}}
\usepackage{longtable,booktabs,array}
\usepackage{calc} % for calculating minipage widths
% Correct order of tables after \paragraph or \subparagraph
\usepackage{etoolbox}
\makeatletter
\patchcmd\longtable{\par}{\if@noskipsec\mbox{}\fi\par}{}{}
\makeatother
% Allow footnotes in longtable head/foot
\IfFileExists{footnotehyper.sty}{\usepackage{footnotehyper}}{\usepackage{footnote}}
\makesavenoteenv{longtable}
\usepackage{graphicx}
\makeatletter
\newsavebox\pandoc@box
\newcommand*\pandocbounded[1]{% scales image to fit in text height/width
  \sbox\pandoc@box{#1}%
  \Gscale@div\@tempa{\textheight}{\dimexpr\ht\pandoc@box+\dp\pandoc@box\relax}%
  \Gscale@div\@tempb{\linewidth}{\wd\pandoc@box}%
  \ifdim\@tempb\p@<\@tempa\p@\let\@tempa\@tempb\fi% select the smaller of both
  \ifdim\@tempa\p@<\p@\scalebox{\@tempa}{\usebox\pandoc@box}%
  \else\usebox{\pandoc@box}%
  \fi%
}
% Set default figure placement to htbp
\def\fps@figure{htbp}
\makeatother
\setlength{\emergencystretch}{3em} % prevent overfull lines
\providecommand{\tightlist}{%
  \setlength{\itemsep}{0pt}\setlength{\parskip}{0pt}}
\usepackage[]{natbib}
\bibliographystyle{plainnat}
\usepackage{booktabs}
\usepackage{bookmark}
\IfFileExists{xurl.sty}{\usepackage{xurl}}{} % add URL line breaks if available
\urlstyle{same}
\hypersetup{
  pdftitle={Functinal Analaysis},
  pdfauthor={Ashan De Silva},
  hidelinks,
  pdfcreator={LaTeX via pandoc}}

\title{Functinal Analaysis}
\author{Ashan De Silva}
\date{2026-01-15}

\usepackage{amsthm}
\newtheorem{theorem}{Theorem}[chapter]
\newtheorem{lemma}{Lemma}[chapter]
\newtheorem{corollary}{Corollary}[chapter]
\newtheorem{proposition}{Proposition}[chapter]
\newtheorem{conjecture}{Conjecture}[chapter]
\theoremstyle{definition}
\newtheorem{definition}{Definition}[chapter]
\theoremstyle{definition}
\newtheorem{example}{Example}[chapter]
\theoremstyle{definition}
\newtheorem{exercise}{Exercise}[chapter]
\theoremstyle{definition}
\newtheorem{hypothesis}{Hypothesis}[chapter]
\theoremstyle{remark}
\newtheorem*{remark}{Remark}
\newtheorem*{solution}{Solution}
\begin{document}
\maketitle

{
\setcounter{tocdepth}{1}
\tableofcontents
}
\chapter{Banach space}\label{banach-space}

\section{Lebesgue spaces}\label{lebesgue-spaces}

\subsection{a}\label{a}

\begin{definition}
\protect\hypertarget{def:unnamed-chunk-1}{}\label{def:unnamed-chunk-1}

markdown
Let \(X\) be a set. A \textbf{\(\sigma\)-algebra} \(\mathcal{I}\) on \(X\) is a collection of subsets of \(X\) such that:

\begin{enumerate}
\def\labelenumi{\arabic{enumi}.}
\tightlist
\item
  \(\emptyset \in \mathcal{I}\),
\item
  if \(E \in \mathcal{I}\), then \(X \setminus E \in \mathcal{I}\),
\item
  if \(E_n \in \mathcal{I}\) for every \(n \ge 1\), then\\
  \[
      \bigcup_{n=1}^{\infty} E_n \in \mathcal{I}.
  \]
\end{enumerate}

\end{definition}

\begin{itemize}
\tightlist
\item
  Elements of \(\mathcal{I}\) are called \(\mathcal{I}\)-measurable sets,
\item
  \((X,\mathcal{I})\) is a measurable space.
\end{itemize}

\begin{definition}
\protect\hypertarget{def:unnamed-chunk-2}{}\label{def:unnamed-chunk-2}A function \(f : X \to \mathbb{C}\) is said to be measurable if \[
f^{-1}\bigl(\{\, z \in \mathbb{C} : |z - a| < \delta \,\}\bigr) \in \mathcal{T}\] for every \(\delta > 0\) and \(a \in \mathbb{C}\).
\end{definition}

\begin{definition}
\protect\hypertarget{def:unnamed-chunk-3}{}\label{def:unnamed-chunk-3}A (positive) measure is a function

\[
\mu : \mathcal{T} \to [0,\infty]
\]

which is countably additive, in the sense that if \(\{E_n\}_{n=1}^\infty\) is a countable collection of
disjoint measurable sets, then

\[
\mu\!\left( \bigcup_{n=1}^{\infty} E_n \right)
= \sum_{n=1}^{\infty} \mu(E_n).
\]
\end{definition}

\begin{itemize}
\tightlist
\item
  The triple \((X, \mathcal{T}, \mu)\) is called a \emph{measure space}.
\end{itemize}

\textbf{Notation} :

\begin{itemize}
\tightlist
\item
  Let \(0 < p < \infty\),
  \[\mathcal{L}^p(X,\mathcal{I},\mu):=\left\{f:X\rightarrow \mathcal{C}:f \text{ is measurable and }\int_X |f|^p d\mu<\infty\right\}\]

  \begin{itemize}
  \tightlist
  \item
    Such functions are said to be \textbf{\(p\)-integrable}.
  \item
    \(\mathcal{L}^p\) norm of \(f\) \(= ||f||_p=\left(\int_X|f|^p\, d\mu\right)^\frac{1}{p}\)
  \end{itemize}
\item
  \(p=\infty\)
\end{itemize}

\[
\mathcal{L}^{\infty}(X,\mathcal{T},\mu)
= \left\{
    f : X \to \mathbb{C} \;:\;
    \exists\, M > 0 \text{ such that } |f| < M \ \text{[\(\mu\)]-a.e. on } X
  \right\}.
\]

\begin{itemize}
\tightlist
\item
  Such functions are said to be \textbf{essentially bounded}.
\item
  The essential norm \(=\) \(\mathcal{L}^{\infty}\) norm of \(f=
  \|f\|_{\infty}
  = \inf \left\{ M > 0 : |f| < M \ \text{[\(\mu\)]-a.e. on } X \right\}.\)
  \pandocbounded{\includegraphics[keepaspectratio]{fig/fig (4).png}}
  In this section we use the term \textbf{``norm.''} Strictly speaking, we have not yet verified that the expressions introduced actually satisfy the axioms of a norm. That verification will come later. For now, we use the word ``norm'' informally, with the understanding that its legitimacy will be established in due course.
\end{itemize}

\begin{lemma}
\protect\hypertarget{lem:lemma1.1.1}{}\label{lem:lemma1.1.1}Let \((X,\mathcal{T},\mu)\) be a measure space, let \(0 < p < \infty\), and let
\(f \in \mathcal{L}^{p}(X,\mathcal{T},\mu)\).
Then

\[
\|f\|_{p} = 0
\quad \iff \quad
f(x) = 0 \ \text{for [\(\mu\)]-a.e. } x \in X.
\]
\end{lemma}

\begin{proof}
\pandocbounded{\includegraphics[keepaspectratio]{fig/fig (2).png}}
\pandocbounded{\includegraphics[keepaspectratio]{fig/fig (3).png}}
\end{proof}

\textbf{Fact}: \(\lambda \in \mathbb{C}, f\in \mathcal{L}^p(X,I,\mu), 0<p\geq\infty,||\lambda f||_p=|\lambda| ||f||_p\).

\begin{proof}
\pandocbounded{\includegraphics[keepaspectratio]{fig/fig (5).png}}
\end{proof}

\begin{lemma}
\protect\hypertarget{lem:unnamed-chunk-6}{}\label{lem:unnamed-chunk-6}Let \((X,\mathcal{T},\mu)\) be a measure space and let
\(f \in \mathcal{L}^{\infty}(X,\mathcal{T},\mu)\).
Then, for
\[
|f(x)| \le \|f\|_{\infty} [\mu]-\text{a.e.} x \in X.
\]
\end{lemma}

\pandocbounded{\includegraphics[keepaspectratio]{fig/fig (1).png}}

\textbf{Result} : If \(f,g\in \mathcal{L}^P(X,I,\mu)\) then \(f+g\in \mathcal{L}^P(X,I,\mu)\)

\begin{longtable}[]{@{}cc@{}}
\toprule\noalign{}
\(1\leq p <\infty\) & \(p=\infty\) \\
\midrule\noalign{}
\endhead
\bottomrule\noalign{}
\endlastfoot
\pandocbounded{\includegraphics[keepaspectratio]{fig/fig (7).png}} & \pandocbounded{\includegraphics[keepaspectratio]{fig/fig (6).png}} \\
\end{longtable}

\begin{lemma}[Young's inequality]
\protect\hypertarget{lem:unnamed-chunk-7}{}\label{lem:unnamed-chunk-7}Let \(a,b \ge 0\) and \(1 < p < \infty\). Let \(q\) be the conjugate exponent, i.e.

\(\frac{1}{p} + \frac{1}{q} = 1.\)

Then

\[
ab \le \frac{a^{p}}{p} + \frac{b^{q}}{q}.
\]
\end{lemma}

\pandocbounded{\includegraphics[keepaspectratio]{fig/fig (8).png}}
\pandocbounded{\includegraphics[keepaspectratio]{fig/fig (9).png}}

\begin{theorem}[Holder's inequality]
\protect\hypertarget{thm:unnamed-chunk-8}{}\label{thm:unnamed-chunk-8}Fix \(1 \le p < \infty\) and let \(q\) be the conjugate exponent, i.e.

\(\frac{1}{p} + \frac{1}{q} = 1.\)

Let \(f,g : X \to \mathbb{C}\) be measurable functions. Then

\[
\int_X |f g| \, d\mu
\;\le\;
\left( \int_X |f|^{p} \, d\mu \right)^{1/p}
\left( \int_X |g|^{q} \, d\mu \right)^{1/q}=||f||_p ||f||_q.
\]
\end{theorem}

\begin{proof}
\pandocbounded{\includegraphics[keepaspectratio]{fig/fig (10).png}}
\pandocbounded{\includegraphics[keepaspectratio]{fig/fig (11).png}}
\end{proof}

\begin{remark}
If \(p=2\) then \(q=2\) then Holder ineqaulty becomes Cauchy -Schawrz inequlity.
\end{remark}

\begin{theorem}[Minkowski's Inequality]
\protect\hypertarget{thm:mink}{}\label{thm:mink}Fix \(1 \le p \le \infty\). Let \(f,g : X \to \mathbb{C}\) be measurable functions. Then
\[
\|f + g\|_{p} \le \|f\|_{p} + \|g\|_{p}.
\]
\end{theorem}

\begin{proof}
\pandocbounded{\includegraphics[keepaspectratio]{fig/mikwoski-1.png}}
\pandocbounded{\includegraphics[keepaspectratio]{fig/mikwoski-2.png}}
\pandocbounded{\includegraphics[keepaspectratio]{fig/mikwoski-3.png}}
\pandocbounded{\includegraphics[keepaspectratio]{fig/mikwoski-4.png}}
\end{proof}

Next, we consider the following question:

\textbf{Question} : For which measurable functions \(f : X \to \mathbb{C}\) do we have \(\|f\|_{p} = 0\)?

\textbf{Answer}: By lemma @ref(lemm:lemma1.1.1) \(||f||_p=0 \iff f=0 ~[\mu] -\) a.e. Precisely those functions such that \(f(x) = 0\) for \(\mu\)-almost every \(x \in X\).

In particular, there are some functions \(f\) which are not identically zero but have zero \(\mathcal{L}^{p}\)-norm. This is unfortunate, so we typically consider the following quotient space:

We define

\[
{L}^{p}(X,\mathcal{T},\mu)
    = \frac{\mathcal{L}^{p}(X,\mathcal{T},\mu)}{N_{p}},
\]

where

\[
N_{p}
    = \{\, f \in \mathcal{L}^{p}(X,\mathcal{T},\mu) : \|f\|_{p} = 0 \,\}.
\]

We have seen that for any \(\lambda \in \mathbb{C}\) and any
\(f, g \in \mathcal{L}^{p}(X,\mathcal{T},\mu)\), we always have

\begin{align*}
\|\lambda f\|_{p} &= |\lambda|\,\|f\|_{p},\\
\|f + g\|_{p} &\le \|f\|_{p} + \|g\|_{p}.\\
& \uparrow\\
&\text{ By Mink}\\
\end{align*}

\textbf{Claim}: \(\mathcal{L}^p\) is vector space over \(\mathbb{C}\).

\begin{proof}
\leavevmode

\begin{enumerate}
\def\labelenumi{\arabic{enumi}.}
\tightlist
\item
  Zero function
\end{enumerate}

Let \(0(x) := 0\) for all \(x\). Then \(|0|^p = 0\) and \(\int_X |0|^p\,d\mu = 0 < \infty,\) so \(0 \in \mathcal{L}^p\).

\begin{enumerate}
\def\labelenumi{\arabic{enumi}.}
\setcounter{enumi}{1}
\tightlist
\item
  Closed under scalar multiplication
\end{enumerate}

Let \(f \in \mathcal{L}^p\) and \(\lambda \in \mathbb{C}\). Then
\[
|\lambda f|^p = |\lambda|^p |f|^p,
\]
so
\[
\int_X |\lambda f|^p\,d\mu
= |\lambda|^p \int_X |f|^p\,d\mu < \infty.
\]
Thus \(\lambda f \in \mathcal{L}^p\).

\begin{enumerate}
\def\labelenumi{\arabic{enumi}.}
\setcounter{enumi}{2}
\tightlist
\item
  Closed under addition
\end{enumerate}

Let \(f,g \in \mathcal{L}^p\). Use the standard inequality for \(p \ge 1\):
\[
|f+g|^p \le 2^{p-1}\big(|f|^p + |g|^p\big).
\]
Integrate:
\[
\int_X |f+g|^p\,d\mu
\le 2^{p-1} \left( \int_X |f|^p\,d\mu + \int_X |g|^p\,d\mu \right)
< \infty,
\]
since both integrals on the right are finite. Hence \(f+g \in \mathcal{L}^p\).

\begin{enumerate}
\def\labelenumi{\arabic{enumi}.}
\setcounter{enumi}{3}
\tightlist
\item
  Vector space axioms
\end{enumerate}

The pointwise operations
\[
(f+g)(x) := f(x)+g(x), \quad (\lambda f)(x) := \lambda f(x)
\]
inherit associativity, commutativity, distributivity, etc., from \(\mathbb{C}\). Together with steps 1--3, this shows \(\mathcal{L}^p(X,\mathcal{T},\mu)\) is a vector space over \(\mathbb{C}\).

Then
\[
L^{p}(X,\mathcal{T},\mu)
= \frac{\mathcal{L}^{p}(X,\mathcal{T},\mu)}{N_{p}}
\]
is the quotient of this vector space by the subspace \(N_p\), so it is also a vector space.

\end{proof}

\textbf{Claim}: \(N^p\) is subspace of \(\mathcal{L}^p\)

\begin{proof}
Let \(f,g\in N^p\) and \(\lambda\in \mathbb{C}\),

\begin{itemize}
\tightlist
\item
  \(0_{map}\in N^p\implies N^p \neq \emptyset\)
\item
  \(\|\lambda f\|_{p} = |\lambda|\,\|f\|_{p}=0\)
\item
  \(\|f + g\|_{p} \le \|f\|_{p} + \|g\|_{p}=0 \implies \|f + g\|_{p}=0.\)
\end{itemize}

Thus, \(N^p\) is a subspace of \(\mathcal{L}^p\).
\end{proof}

Thus, \(L^p\) is subspace. Hence, \(N_{p}\) is a subspace of \(L^{p}(X,\mathcal{T},\mu)\); therefore
\(L^{p}(X,\mathcal{T},\mu)\) is a vector space over \(\mathbb{C}\).

If for \(f \in L^{p}(X,\mathcal{T},\mu)\) we denote by \([f]\) its image
in the quotient space \(L^{p}(X,\mathcal{T},\mu)\), then

\[
\lambda [f] + [g] = [\,\lambda f + g\,].
\]

Define
\[\|[f]\|_p=\|f\|_p\]
More ever, \(\|[\cdot]\|_p\) well defined.

\begin{proof}
Let \(f,g\in \mathcal{L}^p\)
By Minkowski's inequlirty we can get,

\begin{equation}
\bigl|\,\|f\|_{p} - \|g\|_{p}\,\bigr|
    \;\le\;
    \|\,f - g\,\|_{p},
    \qquad f,g \in L^{p}(X,\mathcal{T},\mu).
\end{equation}

Suppos that \([f]=[g]\).
Then, \(f-g\in N^p\implies f-g\in N^p\). Then \(\|\,f - g\,\|_{p}=0\). Thus,

\begin{align*}\bigl|\,\|f\|_{p} - \|g\|_{p}\,\bigr|
    \;\le\;
    \|\,f - g\,\|_{p}=0 
    & \implies \|[f-g] \|_p=0\\ 
    &\implies \bigl|\,\|f\|_{p} - \|g\|_{p}\,\bigr|=0\\
    &\implies \,\|f\|_{p} = \|g\|_{p}
\end{align*}
\end{proof}

Note that \(\|[f]\|_p = 0\) if and only \([f] = 0_{L^p}\) in \(L^p(X,I,\mu)\).

\begin{proof}
\leavevmode

\begin{itemize}
\tightlist
\item
  \(\mathbf{\implies}\) :\\
  \begin{align*}  
  \|[f]\|_p = 0 &\implies \|f\|_p = 0.\\
  &\implies f\in N^p \\
  & \implies [f] = [0] = 0_{L^p}\\
  \end{align*}
\item
  \(\mathbf{\Longleftarrow}\) :
  \begin{align*}
   [f]=0_{L^p} & \implies [f]=[0]\\
  & \implies f-0\in N^p\\
  & \implies f\in N^p \\
  & \implies \|f\|_p=0\\
  & \implies \|[f]\|_p = \|f\|_p = 0.
  \end{align*}
\end{itemize}

\end{proof}

Now we can avoid the problem that have earlier. Now we can defnie the norm.

Here's a clean, well‑structured Markdown version of your text, with mathematical expressions formatted clearly and consistently.

\begin{center}\rule{0.5\linewidth}{0.5pt}\end{center}

\section{A point of notation}\label{a-point-of-notation}

For convenience, mathematicians agree to write \(f\) instead of \([f]\). This causes very little confusion; the only thing to keep in mind is that one can capture the behaviour of an element in \(L^p(X,\mathcal{T},\mu)\) only up to sets of zero \(\mu\)-measure. For the rest of this course, we will use this convention and write elements of the quotient space \(L^p\) simply as functions.

\subsection{Summary}\label{summary}

For \(1 \le p \le \infty\):

\begin{enumerate}
\def\labelenumi{\arabic{enumi}.}
\item
  \textbf{Vector space:} \(L^p(X,\mathcal{T},\mu)\) is a vector space over \(\mathbb{C}\).
\item
  \textbf{Definition of the \(p\)-norm:} To every \(f \in L^p(X,\mathcal{T},\mu)\) we associate a non‑negative number defined by\\
  \[
  \|f\|_p = \left( \int_X |f|^p \, d\mu \right)^{1/p}, \qquad 1 \le p < \infty,
  \]
  and for \(p = \infty\),
  \[
  \|f\|_\infty = \inf\{ M \ge 0 : |f(x)| \le M \text{ for almost every } x \}.
  \]
\item
  \textbf{Homogeneity:} For every \(\lambda \in \mathbb{C}\) and \(f \in L^p(X,\mathcal{T},\mu)\),
  \[
  \|\lambda f\|_p = |\lambda|\, \|f\|_p.
  \]
\item
  \textbf{Triangle inequality:} For every \(f,g \in L^p(X,\mathcal{T},\mu)\),
  \[
  \|f + g\|_p \le \|f\|_p + \|g\|_p.
  \]
\item
  \textbf{Definiteness:} For every \(f \in L^p(X,\mathcal{T},\mu)\),
  \[
  \|f\|_p \ge 0,
  \]
  with equality if and only if \(f = 0\) almost everywhere.
\end{enumerate}

Properties (iii), (iv), and (v) show that \(\|\cdot\|_p\) defines a norm, so \(L^p(X,\mathcal{I},\mu)\) is a \textbf{normed linear space}.

\begin{definition}
\protect\hypertarget{def:unnamed-chunk-16}{}\label{def:unnamed-chunk-16}\textbf{Banach spaces} are normed linear spaces with an additional property: they are \emph{complete}, meaning every Cauchy sequence converges.
\end{definition}

Our next task is to show that \(L^p(X,\mathcal{I},\mu)\) is a Bannch space. (We need to show that complete space)

Before doing so, we recall some important results from measure theory.

\begin{lemma}[Chebyshev’s Inequality]
\protect\hypertarget{lem:unnamed-chunk-17}{}\label{lem:unnamed-chunk-17}Let \((X,\mathcal{T},\mu)\) be a measure space and let \(f\) be a non‑negative measurable function on \(X\).\\
Then, for every \(\lambda > 0\),
\[
\mu\{x \in X : f(x) \ge \lambda\}
\;\le\;
\frac{1}{\lambda} \int_X f \, d\mu.
\]
\end{lemma}

\begin{proof}
Let \(E_\lambda=\{x \in X : f(x) \ge \lambda\}\), Then,
\[
\int_X f \,d\mu \geq \int_{E_\lambda} f \,d\mu  \ge \int_{E_\lambda} \lambda \,d\mu =\lambda \int_{E_\lambda}  \,d\mu =\lambda \mu(E_\lambda)
\]
\end{proof}

\begin{lemma}[Borel–Cantelli Lemma]
\protect\hypertarget{lem:unnamed-chunk-19}{}\label{lem:unnamed-chunk-19}Let \((X,\mathcal{T},\mu)\) be a measure space and let \(\{E_n\}_{n=1}^\infty\) be a collection of measurable sets such that \(\sum_{n=1}^\infty \mu(E_n) < \infty.\)
Then \(\mu\)-almost every \(x \in X\) belongs to at most finitely many of the sets \(E_n\).
\end{lemma}

\begin{proof}
\begin{align*}
S&:=\left\{x\in X: x \text{  belongs to infinitly many }E_n\right\}\\
&=\bigcap_{N=1}^\infty \bigcup_{k=N}^\infty E_k (I will explain this later)
\end{align*}

\begin{align*}
\mu(S) &\le \mu \left(\cup_{k=N}^\infty\right)\\
& \le \sum_{k=N}^\infty \mu(E_k) \text{ for all } N
\end{align*}
Then left hand side is goes to zero as \(N\to \infty\).

\pandocbounded{\includegraphics[keepaspectratio]{fig/fig (11).png}}
\end{proof}

\begin{lemma}[Fatou’s Lemma]
\protect\hypertarget{lem:unnamed-chunk-21}{}\label{lem:unnamed-chunk-21}Let \((X,\mathcal{T},\mu)\) be a measure space and let \(\{f_n\}_{n=1}^\infty\) be a sequence of non‑negative measurable functions on \(X\). Then,
\[
\int_X \liminf_{n\to\infty} f_n \, d\mu
\;\le\;
\liminf_{n\to\infty} \int_X f_n \, d\mu.
\]
\end{lemma}

\begin{lemma}[A Technical Convergence Lemma]
\protect\hypertarget{lem:unnamed-chunk-22}{}\label{lem:unnamed-chunk-22}

Let \((X,\mathcal{T},\mu)\) be a measure space and let \(1 \le p \le \infty\).\\
Let \(\{f_n\}_{n=1}^\infty \subset L^p(X,\mathcal{T},\mu)\) be a sequence such that there exists a sequence of positive numbers \(\{\varepsilon_n\}_{n=1}^\infty\) with
\[
\sum_{n=1}^\infty \varepsilon_n < \infty,
\]
and
\[
\|f_n - f_{n+1}\|_p \le \varepsilon_n^2, \qquad n \ge 1.
\]

Then there exists \(f \in L^p(X,\mathcal{T},\mu)\) such that

\begin{itemize}
\item
  pointwise a.e. convergence:
  \[
  \lim_{n\to\infty} f_n(x) = f(x) \quad \text{for } \mu\text{-almost every } x \in X,
  \]
\item
  convergence in \(L^p\):
  \[
  \lim_{n\to\infty} \|f - f_n\|_p = 0.
  \]
\end{itemize}

\end{lemma}

\chapter{Hello bookdown}\label{hello-bookdown}

All chapters start with a first-level heading followed by your chapter title, like the line above. There should be only one first-level heading (\texttt{\#}) per .Rmd file.

\section{A section}\label{a-section}

All chapter sections start with a second-level (\texttt{\#\#}) or higher heading followed by your section title, like the sections above and below here. You can have as many as you want within a chapter.

\subsection*{An unnumbered section}\label{an-unnumbered-section}
\addcontentsline{toc}{subsection}{An unnumbered section}

Chapters and sections are numbered by default. To un-number a heading, add a \texttt{\{.unnumbered\}} or the shorter \texttt{\{-\}} at the end of the heading, like in this section.

\chapter{Cross-references}\label{cross}

Cross-references make it easier for your readers to find and link to elements in your book.

\section{Chapters and sub-chapters}\label{chapters-and-sub-chapters}

There are two steps to cross-reference any heading:

\begin{enumerate}
\def\labelenumi{\arabic{enumi}.}
\tightlist
\item
  Label the heading: \texttt{\#\ Hello\ world\ \{\#nice-label\}}.

  \begin{itemize}
  \tightlist
  \item
    Leave the label off if you like the automated heading generated based on your heading title: for example, \texttt{\#\ Hello\ world} = \texttt{\#\ Hello\ world\ \{\#hello-world\}}.
  \item
    To label an un-numbered heading, use: \texttt{\#\ Hello\ world\ \{-\#nice-label\}} or \texttt{\{\#\ Hello\ world\ .unnumbered\}}.
  \end{itemize}
\item
  Next, reference the labeled heading anywhere in the text using \texttt{\textbackslash{}@ref(nice-label)}; for example, please see Chapter \ref{cross}.

  \begin{itemize}
  \tightlist
  \item
    If you prefer text as the link instead of a numbered reference use: \hyperref[cross]{any text you want can go here}.
  \end{itemize}
\end{enumerate}

\section{Captioned figures and tables}\label{captioned-figures-and-tables}

Figures and tables \emph{with captions} can also be cross-referenced from elsewhere in your book using \texttt{\textbackslash{}@ref(fig:chunk-label)} and \texttt{\textbackslash{}@ref(tab:chunk-label)}, respectively.

See Figure \ref{fig:nice-fig}.

\begin{Shaded}
\begin{Highlighting}[]
\FunctionTok{par}\NormalTok{(}\AttributeTok{mar =} \FunctionTok{c}\NormalTok{(}\DecValTok{4}\NormalTok{, }\DecValTok{4}\NormalTok{, .}\DecValTok{1}\NormalTok{, .}\DecValTok{1}\NormalTok{))}
\FunctionTok{plot}\NormalTok{(pressure, }\AttributeTok{type =} \StringTok{\textquotesingle{}b\textquotesingle{}}\NormalTok{, }\AttributeTok{pch =} \DecValTok{19}\NormalTok{)}
\end{Highlighting}
\end{Shaded}

\begin{figure}

{\centering \includegraphics[width=0.8\linewidth,alt={Plot with connected points showing that vapor pressure of mercury increases exponentially as temperature increases.}]{_main_files/figure-latex/nice-fig-1} 

}

\caption{Here is a nice figure!}\label{fig:nice-fig}
\end{figure}

Don't miss Table \ref{tab:nice-tab}.

\begin{Shaded}
\begin{Highlighting}[]
\NormalTok{knitr}\SpecialCharTok{::}\FunctionTok{kable}\NormalTok{(}
  \FunctionTok{head}\NormalTok{(pressure, }\DecValTok{10}\NormalTok{), }\AttributeTok{caption =} \StringTok{\textquotesingle{}Here is a nice table!\textquotesingle{}}\NormalTok{,}
  \AttributeTok{booktabs =} \ConstantTok{TRUE}
\NormalTok{)}
\end{Highlighting}
\end{Shaded}

\begin{table}

\caption{\label{tab:nice-tab}Here is a nice table!}
\centering
\begin{tabular}[t]{rr}
\toprule
temperature & pressure\\
\midrule
0 & 0.0002\\
20 & 0.0012\\
40 & 0.0060\\
60 & 0.0300\\
80 & 0.0900\\
\addlinespace
100 & 0.2700\\
120 & 0.7500\\
140 & 1.8500\\
160 & 4.2000\\
180 & 8.8000\\
\bottomrule
\end{tabular}
\end{table}

\chapter{Parts}\label{parts}

You can add parts to organize one or more book chapters together. Parts can be inserted at the top of an .Rmd file, before the first-level chapter heading in that same file.

Add a numbered part: \texttt{\#\ (PART)\ Act\ one\ \{-\}} (followed by \texttt{\#\ A\ chapter})

Add an unnumbered part: \texttt{\#\ (PART\textbackslash{}*)\ Act\ one\ \{-\}} (followed by \texttt{\#\ A\ chapter})

Add an appendix as a special kind of un-numbered part: \texttt{\#\ (APPENDIX)\ Other\ stuff\ \{-\}} (followed by \texttt{\#\ A\ chapter}). Chapters in an appendix are prepended with letters instead of numbers.

\chapter{Footnotes and citations}\label{footnotes-and-citations}

\section{Footnotes}\label{footnotes}

Footnotes are put inside the square brackets after a caret \texttt{\^{}{[}{]}}. Like this one \footnote{This is a footnote.}.

\section{Citations}\label{citations}

Reference items in your bibliography file(s) using \texttt{@key}.

For example, we are using the \textbf{bookdown} package \citep{R-bookdown} (check out the last code chunk in index.Rmd to see how this citation key was added) in this sample book, which was built on top of R Markdown and \textbf{knitr} \citep{xie2015} (this citation was added manually in an external file book.bib).
Note that the \texttt{.bib} files need to be listed in the index.Rmd with the YAML \texttt{bibliography} key.

The RStudio Visual Markdown Editor can also make it easier to insert citations: \url{https://rstudio.github.io/visual-markdown-editing/\#/citations}

\chapter{Blocks}\label{blocks}

\section{Equations}\label{equations}

Here is an equation.

\begin{equation} 
  f\left(k\right) = \binom{n}{k} p^k\left(1-p\right)^{n-k}
  \label{eq:binom}
\end{equation}

You may refer to using \texttt{\textbackslash{}@ref(eq:binom)}, like see Equation \eqref{eq:binom}.

\section{Theorems and proofs}\label{theorems-and-proofs}

Labeled theorems can be referenced in text using \texttt{\textbackslash{}@ref(thm:tri)}, for example, check out this smart theorem \ref{thm:tri}.

\begin{theorem}
\protect\hypertarget{thm:tri}{}\label{thm:tri}For a right triangle, if \(c\) denotes the \emph{length} of the hypotenuse
and \(a\) and \(b\) denote the lengths of the \textbf{other} two sides, we have
\[a^2 + b^2 = c^2\]
\end{theorem}

Read more here \url{https://bookdown.org/yihui/bookdown/markdown-extensions-by-bookdown.html}.

\section{Callout blocks}\label{callout-blocks}

The R Markdown Cookbook provides more help on how to use custom blocks to design your own callouts: \url{https://bookdown.org/yihui/rmarkdown-cookbook/custom-blocks.html}

\chapter{Sharing your book}\label{sharing-your-book}

\section{Publishing}\label{publishing}

HTML books can be published online, see: \url{https://bookdown.org/yihui/bookdown/publishing.html}

\section{404 pages}\label{pages}

By default, users will be directed to a 404 page if they try to access a webpage that cannot be found. If you'd like to customize your 404 page instead of using the default, you may add either a \texttt{\_404.Rmd} or \texttt{\_404.md} file to your project root and use code and/or Markdown syntax.

\section{Metadata for sharing}\label{metadata-for-sharing}

Bookdown HTML books will provide HTML metadata for social sharing on platforms like Twitter, Facebook, and LinkedIn, using information you provide in the \texttt{index.Rmd} YAML. To setup, set the \texttt{url} for your book and the path to your \texttt{cover-image} file. Your book's \texttt{title} and \texttt{description} are also used.

This \texttt{gitbook} uses the same social sharing data across all chapters in your book- all links shared will look the same.

Specify your book's source repository on GitHub using the \texttt{edit} key under the configuration options in the \texttt{\_output.yml} file, which allows users to suggest an edit by linking to a chapter's source file.

Read more about the features of this output format here:

\url{https://pkgs.rstudio.com/bookdown/reference/gitbook.html}

Or use:

\begin{Shaded}
\begin{Highlighting}[]
\NormalTok{?bookdown}\SpecialCharTok{::}\NormalTok{gitbook}
\end{Highlighting}
\end{Shaded}


\end{document}
